% This file should be replaced with your file with thesis content.
%=========================================================================
% Authors: Michal Bidlo, Bohuslav Křena, Jaroslav Dytrych, Petr Veigend and Adam Herout 2019

% For compilation piecewise (see projekt.tex), it is necessary to uncomment it and change
% \documentclass[../projekt.tex]{subfiles}
% \begin{document}

\chapter{Introduction}\label{ch:introduction}

The rapid advancement and widespread deployment of large language models (LLMs) such as GPT-5.1, Claude Sonnet 4.5, Gemini 3 and others have transformed natural-language interaction with computers.
These models power chatbots, code assistants, translation services, and creative tools used daily by millions of users.

However, their remarkable capabilities come with significant safety and ethical risks.
LLMs can generate harmful, biased, misleading, or illegal content when subjected to carefully crafted adversarial inputs, a practice commonly known as \emph{jailbreaking}~\cite{wei2023jailbroken, zou2023universal}.

Real-world incidents such as ChatGPT being tricked into providing bomb-making instructions~\cite{yaser2023potential}, or Gemini's image-generation controversy~\cite{andrew2025agonistic}, have demonstrated that even flagship commercial models remain vulnerable.
In response, red teaming, a cybersecurity technique involving simulated attacks to expose vulnerabilities, has been adopted by leading AI organisations (OpenAI, Anthropic, Google DeepMind, Meta AI) as a core component of LLM safety evaluation~\cite{openai2024redteaming}.

With the adoption of the EU AI Act in 2024, systematic risk assessment including red teaming will become a legal requirement for high-risk AI systems deployed in the European Union from 2026 onward~\cite{eu2024aiact}.
Consequently, efficient, reproducible, and extensible red-teaming tools are no longer a luxury but an essential part of responsible AI development as it will be a legal requirement in the future.

Despite significant progress, most existing open-source red-teaming frameworks suffer from limited modularity, poor support for modern systems, query cost and computational requirements, etc., that hinder adoption by smaller research teams and individual developers~\cite{purpur2025building, schoepf2025madmax, belaire2025automatic, munoz2024pyrit}.
This creates a clear need for a new, lightweight, developer-friendly red-teaming toolkit that lowers the barrier to LLM systematic safety testing.

The main goal of this bachelor's thesis is therefore the design, implementation, and evaluation of a modular open-source red-teaming toolkit for large language models that addresses the identified shortcomings of current solutions.

%=========================================================================

% For compilation piecewise (see projekt.tex), it is necessary to uncomment it
% \end{document}